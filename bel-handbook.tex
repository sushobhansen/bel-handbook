\documentclass[12pt]{article}

\usepackage{amsmath}
\usepackage{graphicx}

\usepackage{fancyhdr} 
\pagestyle{fancy}
\fancyhf{}
\rhead{Built Environment Lab at IITGN}
\lhead{Student Handbook}
\cfoot{\thepage}

\usepackage{hyperref}
\usepackage[
    type={CC},
    modifier={by-nc-sa},
    version={3.0},
]{doclicense}
\pagenumbering{arabic}

\begin{document}

\begin{titlepage}
	\centering
	\includegraphics[scale=0.075]{BEL-logo.png} \\
	{\LARGE Built Environment Lab at IIT Gandhinagar \\
	\vspace{2cm}
	Handbook for Research Students}
	\vspace{2cm}
	
	\begin{large}	
	\textbf{PI: Dr. Sushobhan Sen} \\
	Assistant Professor of Civil Engineering \\
	IIT Gandhinagar \\
	Office: AB 3/306E \\
	E-mail: sushobhan.sen@iitgn.ac.in \\
	Office Phone (VoIP): 2606

	\vspace{2cm}
	\end{large}
	
	Last Updated: \today
\end{titlepage}

This document is meant as a supplementary guide for research students in the Built Environment Lab at IIT Gandhinagar and is meant for internal use within the lab. All rules and statutes of IIT Gandhinagar or any higher governing authority supersede this document. This document has not been endorsed by IIT Gandhinagar or any higher governing authority. For any questions, please contact the PI.

\doclicenseThis

Copyright \copyright\ \the\year{}, Dr. Sushobhan Sen
\newpage

\tableofcontents
\newpage

\section{Introduction}
The Built Environment Lab at IIT Gandhinagar, within the Discipline of Civil Engineering, conducts research on the interaction between civil infrastructure and the environment. While "infrastructure" is a broad term, the primary focus of the lab is on road infrastructure. While the lab primarily focuses on computational techniques, field and laboratory work is also performed depending on the project. 

The \textbf{objective} of the Built Environment Lab is to train students to conduct \textit{demonstrably} novel inter-disciplinary research in pavement engineering, urban sustainability, and climate change. Here, the term "demonstrably" implies publishing research in high-quality journals as well as presenting it at various national and international conferences, in addition to various research showcases within IIT Gandhinagar, in that order of importance. Thus, research will be judged by peers and wider society, not through any objective assessment ("exam"). 

\subsection{About the PI}
Dr. Sushobhan Sen is an assistant professor in the Discipline of Civil Engineering at IIT Gandhinagar. He joined the institute on December 27, 2022. Prior to that, he was a postdoctoral research associate in the Department of Civil and Environmental Engineering at the University of Pittsburgh, USA from January, 2020 to December, 2022. He received his PhD in Civil Engineering from the University of Illinois at Urbana-Champaign, USA (UIUC) in December, 2019 with a thesis titled, "Role of Pavements in Urban Energetics." He also holds a Master of Science (MS) in Civil Engineering from UIUC (August, 2015) and a Bachelor of Technology (BTech) in Civil Engineering from IIT Roorkee (May, 2013). He has published over two dozen peer-reviewed journal and conference articles in the areas of pavement engineering, urban heat islands, sustainability, and climate change, and has served as a reviewer for over 100 articles in various venues. His interests are primarily in the area of computational mechanics, although he is also experienced with machine learning and pavement materials characterization. He has also served as Co-Director of Communications for the International Society for Concrete Pavements (ISCP). His full CV can be accessed from \href{https://github.com/sushobhansen/CV/blob/master/sushobhan-sen-cv.pdf}{here}.

\subsection{About the Lab's logo}
The Built Environment Lab logo, shown below, consists of three pairs of lines connected by a circle. Each pair of lines represents one of the themes of the Lab viz., pavement engineering, urban sustainability, and climate change. They are depicted as pairs to show the completeness of each theme as a field of study on its own. Notice that the pairs, although tending to converge at a point, never actually meet. This shows that while each theme has implications for society as a whole, they are traditionally studied separately. The circle in between, therefore, indicates the inter-disciplinary focus of the Lab in connecting these themes to produce research that has a significant impact on society. Finally, the overall aesthetic of the logo is that of a wheel with three spokes, moving forward with the aim of the adding new knowledge to the world. The color scheme, black and yellow, is an homage to the city of Pittsburgh, USA, where the PI learned many valuable lessons. 

\begin{figure}[h]
	\centering
	\includegraphics[scale=0.075]{BEL-logo.png}
\end{figure}

The Lab's logo is an important part of its identity. As a member of the Lab, you should use it in scholarly work such as posters and presentations. Please ask the PI for a high-resolution version when you need it.

\subsection{About this handbook}
This handbook is a dynamic document that will be updated regularly over time. It is meant to assist you in conducting your research, and not to be a "rulebook" that has to be strictly followed. Therefore, read this as a helpful guide while you are in the Lab. All regulations of IIT Gandhinagar and higher authorities always apply to you at all times. 

\newpage
\section{Conducting research}
The primary aim of research is to develop new knowledge on top of the existing body of knowledge that exists, albeit in a very narrow and specific field. Inter-disciplinary research typically involves combining knowledge and skills from multiple disciplines, such that the final output is more valuable than the individual pieces may have been within their own respective disciplines. This is easier said than done: mastering one discipline is difficult by itself, mastering more than one is quite a challenge. 

Fortunately, there are some general steps that can be adopted to get started. These are discussed below.

\subsection{Formulating a problem}
The most difficult part of research is to formulate a research problem. A research problem is essentially a hypothesis that is posed: "under certain conditions, a particular result must be true." Research itself then reduces to a systematic procedure to reject the null hypothesis. A good research problem has four characteristics:

\begin{enumerate}
	\item It is novel i.e., it has never been done before
	\item It is a worthwhile problem that adds value to the field as well as to society at large; in other words, it is not trivial
	\item It is negatable i.e., you can formulate a clear null hypothesis that must then be rejected
	\item The systematic work necessary to reject the null hypothesis can be completed such that you complete your thesis on time
\end{enumerate}

Note that points 1-3 create a lower bound on the quality of work, while point 4 sets an upper bound. These are, naturally, qualitative bounds and every student will set a different expectation from themselves. But it provides some guidance on formulating a research problem. 

\textbf{Important:} A subtle point from these guidelines is that you can't get away with "easy" research. While this is inherently subjective, if the results of your hypothesis are obvious, then the work is most likely not worthwhile. Moreover, if the techniques you use for your work are not rigorous, then your results are also questionable. Don't avoid difficult problems - use your PhD to acquire difficult skills and solve difficult problems. Your future will be brighter as a result.

\subsection{Reading papers}
Notice that your research problem needs to be novel. To know whether this is indeed the case, you must know the state of the art in your field. This requires \textit{critically} reading the latest literature, in particular peer-reviewed journal papers, on a regular basis. IIT Gandhinagar subscribes to many top journals, which can be easily accessed when connected to the campus WiFi (see the Library's website for off-campus access, or if you have trouble finding a paper).

A literature review is the first and most basic task that you must perform. Although it is a task, it should also become a habit: you should keep reading literature throughout your time here. Nonetheless, to begin a literature review, a good approach is the so-called \textit{snowball method}, as outlined below:

\begin{enumerate}
	\item To get started, you and the PI will discuss a tentative research topic and generate a couple of keywords. You can then use these keywords to search for some of the latest papers (say 5-10) on \href{https://scholar.google.com/}{Google Scholar}.
	\item Read these papers critically. Initially, it will be a slow process and a lot of things will not make sense - keep going. After you read this first batch of papers, look at the references at the end of them. Identify important papers that you need to read to further understand the ones that you just read.
	\item Now read those papers identified from the list of references. Repeat this process with these new set of papers.
	\item After a couple of iterations, you will find that you keep coming across the same set of papers. Your initial literature review is complete. At this stage, you should be well versed with the basic motivation and problems associated with the research topic.
\end{enumerate}

An important point to keep in mind is to restrict your article search \textbf{only} to good journals (for example, those indexed by Scopus or Web of Science). Generally, Google Scholar search results will be limited to such good journals and you can safely read them. Avoid using regular Google or Bing search while looking for papers, as many poor quality and predatory journals will come up in the results. A common journal that seems to come up in India is called the "International Journal of Research in Science and Technology" - avoid it like the plague! If you are not sure about a particular journal, ask the PI. In general, good journals have a scope that is restricted to a narrow area (e.g., pavement engineering, cities, sustainability, etc.), while bad ones have extraordinarily broad scopes (e.g., all of engineering and science). 

However, the literature review does not end here. So far, you have been looking backwards, trying to see some of the historical work in this area, which will typically be decades old. However, to identify a novel problem, you must be well versed with the most recent developments in the area. For this, refine your search to only the latest papers (say papers published in the past 5 years), and only read those references that are of a similar age. Another useful and important tool is \textbf{Google Scholar Alerts}, in which Google will email you a list of the latest papers published at periodic intervals based on the keywords that you supply. Set up an alert to stay on top of the state of the art.

At the end of this process (although it never really ends), you should:

\begin{enumerate}
	\item Be familiar with the historical work in the area
	\item Be well-versed with the latest work in the area as well as the researchers who carried out the work and their institutions
	\item Identify \textbf{gaps} in the current research that you would be reasonably able to fill with some training in a reasonable amount of time
	\item Prepare a list of journals where researchers in your area generally publish
\end{enumerate}

A term used in the preceding discussion is 'critically reading.' This is not as simple as it seems, and there is a high chance that you have never done it before. To critically read a paper means to read it under the assumption that there is something \textit{wrong} with the results: the methodology is questionable, the model makes some unreasonable or excessively specific assumptions, some key tests have not been performed, the results have some unexpected behaviour , etc. Note that your critique of the paper must be \textbf{specific} and \textbf{reasonably actionable}. Consider the following illustrative example:

\textbf{A bad critique:} "The authors used X and Y predictive variables in their model, but could've used some others too."

\textbf{Explanation:} This is a bad critique for several reasons. Firstly, it is not specific - \textit{which} other variables, specifically, could the authors have used? Secondly, because it is not specific, it is also non-actionable because the authors don't know what other variables you have in mind and why. Thirdly, it does not actually critique the work performed but rather questions the scope of the work. The scope of the work is a separate consideration that is constrained by time, money, capabilities, etc. It's possible that the third variable takes years to collect, it is not reasonable to expect the authors to wait years to publish their work - they have to get their degree too! And finally, the critique does not demonstrate any logic. Any paper is trying to prove a hypothesis (or rather reject a null hypothesis). Are the selected variables sufficient to do so? If so, why is a third variable needed? Why should the authors include more variables if the ones they selected are adequate for what they're trying to demonstrate?

A better version of the above critique is as follows:

\textbf{A good critique:} "The authors used X and Y predictive variables in their model. However, work on similar problems published in some papers (cite the papers) has shown that Z is also an extremely important variable whose absence can significantly skew the results. Therefore, the authors should consider including Z, or at least discuss why its absence does not significantly affect the results in this manuscript."

\textbf{Explanation:} This critique is much better because it identifies a specific problem (the fact that Z is missing), explains the logic behind why it should be included (because of past results), and provides a clear action that the authors can perform (include Z). However, in case the action excessively extends the scope of the work, the critique also provides an alternative but reasonable action that the authors could take instead. 

Notice that writing a good critique requires understanding the methodology and/or results and questioning it. Simply reading the results of a paper and accepting them without question is not critical reading. Developing critical reading, and more generally critical reasoning, skills takes some practice and effort, but is essential for performing high quality research.

A useful technique while reading literature is to take notes (by hand, on a note-taking app, in a Word document, etc.). This helps you process the information better, recall it in the future, and also serves as a quick reference. You are encouraged to take notes about papers you read.

\subsection{Performing research}
At this stage, once you have a good research problem, you have finished the most difficult part of doing research (although you should continue to read the latest papers). The next stage is to formalize a plan of research that will fill in at least some of the most important gaps identified in the literature review. The gaps that are identified are bounded (qualitatively) by two factors:

\begin{enumerate}
	\item \textbf{Lower bound:} It must not be trivial. A common example of this is what may be called parametric analysis. For example, in the literature, work has been performed where 5\% of material A was mixed with 95\% of material B. A parametric analysis would be to increase the dosage of material A. While this is a gap, it is not a very significant one from a scientific and societal perspective. 
	\item \textbf{Upper bound:} The gap must not be so large that it cannot be bridged in a reasonable amount of time. For example, a gap might involve the fact that a certain (important) test has never been performed on some material. You may propose to perform it. However, you do not have access to the equipment needed, or even if you do, it will take 5-6 years to reach the level of mastery needed to use it correctly. In this case, the gap identified is beyond your means to fill.
\end{enumerate}

Between these two bounds, you can identify any gap. Notice an important feature implied by these bounds: you will have to learn some new skills, but in a reasonable amount of time. Unless you came with particularly good skills from your previous education, you will not be able to fill a research gap that meets the lower bound. This is a common problem with research students and it is apt to take it head-on over here. Here is some sagacious advice:

\begin{center}
\large{You are employable because of your skills.\\ A degree is a piece of paper.}
\end{center}

Read it again. It is very tempting to choose a research gap that falls below the lower bound, because it involves minimal effort in terms of acquiring skills and can be relatively fast. Yes, you can get a piece of paper called a degree through this - many people do. And many people are also un- or under-employed. As a student, use this time to learn a difficult skill (within the upper bound), which you can then transfer to your next job. You will not have time in the future to acquire such skills again. Shortcuts taken today will be regretted in the future, when the consequences are much worse. 

That said, you are ultimately responsible for your own thesis. The choice of what gap you will fill is largely yours, but must be of interest to the PI as well. Every attempt will be made to find a gap that both you and the PI are comfortable addressing. However, in case a meeting of the minds is not possible, you may want to pursue your research in another lab where you can receive the support you need. 

Once a suitable research gap is identified, the next step is to come up with a research plan. A good strategy here is to divide and conquer:

\begin{enumerate}
	\item Take a look at the research gap and identify a clear hypothesis from it. This hypothesis yields a null hypothesis, which needs to be disproved.
	\item Divide the null hypothesis into a series of smaller null hypotheses, disproving each of which reinforces the final hypothesis.
	\item Develop a series of activities or tasks that must be performed, in sequential order, to disprove the smaller null hypotheses. 
	\item Identify skills that you need to learn to complete those tasks and how you would learn them (e.g., ask a colleague, take an online course, read a book, etc.) in a reasonable amount of time.
	\item Finally, come up with a tentative timeline to complete these tasks.
\end{enumerate}

It's quite straightforward from here - do the tasks and compile the results. You have successfully performed research! Well, almost. Another major task that is part of your training is reporting research. This will be discussed in more detail in the next chapter.

\subsection{To code or not to code?}
A common problem that you will face is whether to do something manually or write a code for it. This can manifest in several ways. For example, if you are performing experimental work, you will have to process the data - you could do this on Excel, or write code (say, in Python) for it. If you are doing computational work, you could repeatedly write a number of commands to run different processes, or you could write a single script (say, a Bash script) to run them at once. What option should you choose?

The answer to this lies in the question of \textit{how many times} you need to do it and \textit{how long} does it take to do. Remember that research is an \textit{iterative} process. You will almost never do something only once, in fact you are technically never done but just run out of time or funding. Consider the two examples above. You may process some data in one way, and then 6-12 months (or more) later, process it again in a slightly different way. Or, you may have to repeatedly run several processes for your code to find a bug, sometimes months apart as new bugs are discovered. In this case, if you do the work manually, you risk:
\begin{itemize}
	\item Making mistakes, especially if the process involves many steps
	\item Forgetting how to do it in a few months, and having to struggle to remember how to do it again
\end{itemize}

In such situations, you should write code to automate your tasks - \textbf{automation is crucial to research}. This is easier said than done. In all likelihood, you are not used to this, and it will feel like a huge waste of time and effort initially. Consider a qualitative illustration of the workflow shown in Figure \ref{fig:auto-workflow}. This graph shows the cumulative amount of work that you can complete, both initially as well as additional work after several months of break. "Work" here refers to actual research output, which can be used to write a paper. When you use a manual process, you immediately start doing "work" and, under ideal conditions, you continue to do work at a steady pace until you are complete with the initial amount of work that needs to be done. 

\begin{figure}[htbp]
	\centering
	\includegraphics[scale=0.5]{auto-workflow.jpg}
	\caption{A qualitative workflow showing the amount of time needed to do the same amount of work using manual and automated processes}
	\label{fig:auto-workflow}
\end{figure}

When you use an automated process by writing code however, you will not immediately do "work". Rather, a good chunk of time will go towards the initial development of your code. It is during this them when writing code to do your work seems to be a waste of time and effort - why write code when it's not doing any work? However, consider what happens after the initial development is completed. Because the code is running on a computer, which is much faster than a human, the amount of work it does per unit time is much higher. In fact, the same amount of work can be completed in much less time by the automated process than the manual process, but in case of the former, you have to be patient during the initial development. 

The benefits of developing an automated process become even clearer when considering the second iteration of work, which would happen after a break of a few days or months. In this case, under the most ideal of circumstances, the manual process would do work at the same pace as the first iteration. In reality, it would take time to recall the logic of the manual process from a while ago, so there will be a delay. On the other hand, the automated process can be run almost immediately, perhaps with just a slight delay for any additional development that may be needed. This is why a code can be described as "\textbf{stored logic}" - you do not have to recall how you initially developed it in order to use it again, and any additional iterations after the initial development are trivial. Not so with manual processes, where further iterations can be just as difficult as the first one. 

The above illustration does not consider the question of quality: it assumes that, given enough time, both manual and automated processes produce identical work. In reality, manual processes are more prone to human error and may require several iterations of \textit{more} manual processes to ultimately fix. While automated processes are also prone to some degree of error, running multiple iterations after the initial one to fix those errors is trivial. In fact, given the huge amount of time saved, it may be possible to do new work (a new type of analysis on experimental data, a new subroutine in your code, etc.) that can improve the quality of the paper. In conclusion, it is usually a good idea to automate your tasks while performing research.

Why "usually"? There are some situations where it may make more sense to perform a manual task. For example, doing some simple calculations that you know need to be done just once or a few times, such as some initial calculations for a run of your experiments, or counting the size of an array that needs to be allocated in your code. When it's so simple, then the initial development time is much longer than the time it would take to just do it manually, and so it doesn't make sense to write code for it. You should however, use Excel instead of a calculator so that there is a record of what you did, in case you need to go back and check. 

Writing code to automate a process is not simple and needs training and practice. Talk to the PI to get started. 

\subsection{Useful resources}
While every thesis is different, and the Built Environment Lab works on a variety of different topics, there are some commonly-used tools that you may use. A \textit{non-exhaustive} list of tools is provided below:

\begin{enumerate}
	\item \textbf{Operating systems:} You are most likely already familiar with Windows. However, a lot of work done in the Lab will be on Linux, typically Ubuntu (and especially using the terminal). You can also use WSL2 on Windows to get the best of both worlds albeit for smaller problems.
	\item \textbf{Image Processing:} ImageJ
	\item \textbf{Programming languages}
	\begin{enumerate}
		\item \textbf{Scripting languages:} These can be used for simple data manipulation and analysis, or as a wrapper for other tools. The most important one is Python, and MATLAB and Excel VBA may also be useful. For very simple work, just plain Excel is also fine. For Python, Jupyter notebooks is fine for initial work, but ultimately you must learn to write and deploy script files.
		\item \textbf{Compiled languages:} These are used to solve more complex problems, particularly those where speed is a very important constraint. Typically, any numerical model that does not involve using third-party tools will need to be written in a compiled language. The Lab will mainly use C++. FORTRAN and C\# may be used occasionally. Wherever possible, GNU compilers will be used.
		\item \textbf{Parallelization}: For numerical problems where speed is very important, the code written in any compiled language may also need to be parallelized, for example to run on a cluster or supercomputer. The Lab will most commonly use OpenMP and MPI for parallelization, and CUDA may be used occasionally. 
		\item \textbf{Integrated Development Environment:} The Lab prefers to use VS Code.
		\item \textbf{Version control:} The Lab uses Git, with additional use of Github for collaboration. By default, you should set a repository to be private and discuss with the PI before making it public. Public repositories usually have a GPL v2 license. Make sure to document your work in the README file.
	\end{enumerate}
	\item \textbf{Reporting:} Short reports, especially to external agencies, will be in Word. However, papers and theses must strictly be in \LaTeX (like this document). An exception may be made if a collaborator insists otherwise. For grammar and spell checks, you can install Grammarly from the Library.
	\item \textbf{References manager:} The built-in references manager in Word should generally be avoided. Mendeley and EndNote are good choices.
	\item \textbf{Computational Fluid Dynamics:} OpenFOAM non-commercial version (from \href{http://openfoam.org/}{openfoam.org}). We may also use ANSYS or COMSOL, for which the Institute has licenses.
	\item \textbf{Solid Mechanics:} ABAQUS or ANSYS, for which the Institute has licenses, or CalculiX
	\item \textbf{General continuum mechanics}: FEniCS
	\item \textbf{Data visualization:} ParaView or custom Python scripts
	\item \textbf{Geographic Information Systems:} QGIS or custom Python scripts
	\item \textbf{Machine Learning:} Tensorflow and Keras (using Python), PyTorch and Scikit-learn may also be possible
	\item \textbf{Mesh generation:} Gmsh
	\item \textbf{CAD:} FreeCAD
	\item \textbf{Symbolic math:} Mathematica, for which the Institute has licenses.
	\item \textbf{Plotting experimental data:} Origin, for which the Institute has licenses.
\end{enumerate}

Unless otherwise noted, the above tools are generally free. The Lab generally tries to use free software to the maximum extent possible. Note that you will most likely use only small subset of the above tools, depending on your research problem. The only exception is \LaTeX, which everyone will use.

\newpage

\section{Reporting research}

Research is never performed in a vacuum. It needs to be reported in a manner by which it can be judged by peers and society at large, at regular intervals, and the feedback suitably incorporated. In general, there are four ways by which you will report your work:

\begin{enumerate}
	\item Reports, typically written to funding agencies. These will typically be written in Word, in whatever format is specified.
	\item Papers submitted to journals and conferences. These will typically be written in \LaTeX. Journal papers in good journals will be preferred over conference papers.
	\item Posters at conferences or research showcase events organized on campus. PowerPoint and Publisher are good choices for these. Ensure that the Lab's logo is displayed on top of the poster.
	\item Your thesis, which will typically be written in \LaTeX.
\end{enumerate}

Of these, papers are by far the most important, while the thesis is mandatory. They are discussed in greater detail below.

\subsection{Writing and publishing papers}
Publishing research papers in high quality peer-reviewed journals is by far the most important activity you will perform to demonstrate the quality of your research. Quality of papers is more important than quantity however, you are generally expected to publish or submit at least 3 papers before you can graduate. Remember, \textbf{write and submit papers regularly}, don't wait till the end to do so.

The general process of publishing a paper is as follows:

\begin{enumerate}
	\item Write a first draft and send it to the PI. There will be several rounds of review and revision. Other members of the Lab may also be invited to provide an internal review.
	\item Identify a target journal and submit to it. Wait for the review to come back, which can take several months.
	\item Once the review is received, regardless of the decision, set it aside for one day. After that, with a cool head, read the comments. Divide them into three categories: those that can be easily addressed, those that require some effort to address, and those that cannot be addressed.
	\item Address the first two sets of comments and prepare a detailed document (called a Response to Reviewers) detailing how they were addressed (you may need to do some additional work for this as well). In the same document, explain why the third set of comments cannot be addressed.
	\item If the decision was a major revision, then resubmit the paper by the deadline. If it was a rejection, submit it elsewhere.
	\item Repeat the above process till the paper is published.
\end{enumerate}

Note that the PI should be involved at every step of the above process. Authorship will be determined based on the \href{https://www.elsevier.com/authors/policies-and-guidelines/credit-author-statement}{CRediT system}.

At least initially, the most difficult and stressful part of publishing will be actually writing the first draft of the paper. This is normal and part of your training. It is much easier to write a paper if you have a clear hypothesis (and the corresponding null hypothesis) based on the gap identification done during the literature review, as well as all the graphs and data that will go into the manuscript. A paper generally consists of the following sections:

\begin{enumerate}
	\item \textbf{Title:} Make it as specific and short as possible.
	\item \textbf{Abstract:} This is a summary of the paper that must briefly describe the motivation, methodology, and conclusions. An abstract must stand alone i.e., it should make sense without having to read the entire paper, but should invite the reader to read the paper. The abstract is usually 200-300 words long.
	\item \textbf{Introduction:} The introduction is meant to systematically identify the gaps in research that the paper seeks to fill. It must begin with an introduction to the problem with references to older works, followed by a summary of the recent work \textit{with a focus on those parts that help identify the gap in the research}, and finally a paragraph clearly describing the hypothesis addressed in the paper and how it fills the gap in research. \textbf{This final paragraph is the most important one in the entire paper and sets the agenda for it.} It must be written clearly and specifically: the worse this paragraph, the worse will be the rest of the paper. 
	\item \textbf{Methodology:} The paper gets much easier at this point. This section begins with a brief reminder of the hypothesis of the paper and a summary of the methodology adopted. Then, this section is typically divided into subsections that discuss each part of the methodology \textit{in an order such that the work can be reproduced}. For example, if the paper discusses some experimental tests, then sample preparation should be discussed before the tests.
	\item \textbf{Results:} This is the easiest section of all. It merely summarizes the results of the work described in the methodology section, typically in the same order as well (including the division into subsections). This section typically includes lots of graphs and tables, pointing out their most important features that are relevant to the hypothesis. For example, if the hypothesis is that a variable X is highly correlated to another variable Y, then this section should have a scatter plot (with a reference line) and the text should highlight the coefficient of determination (R$^2$) and any other relevant statistics.
	\item \textbf{Discussion} (this is sometimes the last subsection of the Results section): This section combines the results and explains how they are related to the hypothesis. It must be logically shown how, based on the results obtained, one may reasonably infer that the null hypothesis to the original hypothesis posed at the end of the Introduction can be rejected. If this is done successfully, then the paper can be deemed to have filled the identified research gap. This section generally does not have subsections.
	\item \textbf{Conclusion:} This can be thought of as a longer version of the abstract, and must also stand by itself. It should once again summarize the motivation, methodology, and conclusions, clearly showing the contribution of the paper in filling the research gap. Optionally, the limitations of the paper and suggested future work may also be included in the end. Generally, the conclusion should be short and \textbf{not} written in bullet points. This section generally does not have subsections.
\end{enumerate}

Remember that when your paper is reviewed (by the PI, a colleague in the Lab, a collaborator, an external reviewer, etc.), it will be read critically, as you were encouraged to read other papers critically. Merely reporting results is not the point: the results will not be believed at face value and the reviewers will try to find mistakes or questionable assumptions behind the results. Therefore, clearly describing your research gap and methodology is as important, if not more important, than presenting your results. 

\subsection{Writing your thesis}
The unfortunate truth about life is that nobody will ever read your thesis (\textit{especially not} your next employer), but you have to write one to graduate. So, the less time spent on it, the better. There are generally two types of theses:

\begin{enumerate}
	\item \textbf{A book about your research:} As the name suggests, it describes all your research comprehensively and can be quite long. Most importantly, it looks and reads quite differently from your papers.
	\item \textbf{A collection of your papers:} A short introduction, followed by papers you have either published (preferably) or at least submitted, followed by a short conclusion and appendices containing mostly code, figures, data, and graphs.
\end{enumerate}

Needless to say, the second type of thesis is the one you should target, because it involves very little actual writing towards the end of your studies, freeing up time for other things, such as applying for jobs. There is also a second, regulatory benefit: in India, a PhD thesis typically needs to be sent to external reviewers, whose feedback has to be incorporated into your thesis. This is a very slow process than can take 6-12 months, and in this time there is a lot of uncertainty around when you can graduate or whether you maintain your fellowship. This can severely hamper your job search. However, if a majority of your thesis has been published in peer-reviewed journals, then an exception to this rule can be obtained and the delay and uncertainty is reduced. This is also why the mantra specified at the beginning of the section is so important: \textbf{write and submit papers regularly}.

\newpage

\section{Guidelines}
Every degree is unique, but there are some general milestones that every student must meet. For an overview of milestones and procedures, see the relevant advisories on \href{https://sites.google.com/iitgn.ac.in/iws/}{IWS}.

A few more useful tips and guidelines specific to the Lab are mentioned below:

\subsection{Work and professionalism}
\begin{enumerate}
	\item There are no fixed working hours and you can work whenever you want -- early morning, late night, afternoons, etc. If you don't want to work on weekends, that is also fine. However, you are expected to make regular progress in your research, including reading the latest literature, formulating problems, performing experiments, reporting results, publishing papers, etc. If you are having trouble, you should seek out help (from the PI, other members of the lab, other professors, books, etc.) instead of just stopping work. While every effort will be made to provide help, poor progress can lead to dismissal from the programme. 
	\item Once a week, you will meet the PI for an individual meeting of approximately one hour to discuss your progress. Since schedules change every semester, you will need to discuss a mutually convenient time every semester to hold this meeting. During the meet, be ready to show your latest progress. Always make sure to bring some pen and paper (or a tablet, if you prefer that) to take notes and write down any instructions. Do not depend on your memory, it will fail you.
	\item Approximately every two weeks, a group meeting with the entire Lab will be held. You may be asked to present some slides about your work and progress and obtain feedback from the group. This is is meant to be interactive and you are highly encouraged to give feedback to others. This exercise will help you improve your critical thinking and communication skills, aside from potentially helping you overcome any problems you may face during research.
	\item All meetings are compulsory unless you receive an exemption from the PI beforehand. Furthermore, it is \textbf{extremely essential} to come to all meetings on time. Coming late, even by a minute, is a sign that you do not respect the other person's time, and regular tardiness may lead to disciplinary action. \textbf{Come on time.}
	\item An exception to the above rule exists in case of emergencies. In this case, you may skip a meeting or come late without permission. However, you will need to explain your absence as soon as possible and it is expected that such emergency absences will be rare. Note that prolonged absence from campus will require you to apply for leave.
	\item If you need additional meetings, feel free to email the PI. Furthermore, if the PI requires additional assistance from you from time to time, they may email you. While you can work whenever you want, you should generally try to respond to such requests promptly. 
	\item When talking to the PI or other members of the community in the academic area, always be professional and courteous. 
	\item There is no dress code, dress comfortably. If the research setting calls for some safety-related dress code, you must follow it.
\end{enumerate}

\subsection{Coursework}
\begin{enumerate}
	\item Review the relevant advisories to see how many and what type of courses you may be required to take as part of the programme. Generally, courses are your responsibility, as is balancing between the amount of time spent on them and that on research. However, the PI must approve all courses that you sign up for and may suggest some courses that are relevant for your research.
	\item While the primary aim of the programme is to conduct research, do not let your grades in coursework slip either. Refer to the relevant advisory on the minimum CPI that you must maintain.
	\item The Institute also offers some certifications and short-term courses. You are encouraged to explore them.  
\end{enumerate}

\subsection{Well-being}
\begin{enumerate}
	\item The Institute has excellent sports and recreational facilities. You are encouraged to use them. While a PhD certainly involves a lot of work, it's a good idea to have a hobby and some interests to help you relieve stress.
	\item The Institute also has a counselling cell that can help you with any mental-health related issues. Please do not hesitate to use their services if you need to.
\end{enumerate}

\end{document}